\documentclass[a4paper]{article}

\usepackage[utf8]{inputenc}
\usepackage[T1]{fontenc,url}
\usepackage{cite}
\usepackage{hyperref}
\usepackage{amsmath, amssymb}
\usepackage{tikz}
\usepackage{graphicx}
%\usepackage{subcaption}
\usepackage{parskip}
\usepackage{lmodern}
\usepackage{algorithm}
\usepackage{algpseudocode}
\usepackage{epigraph}
\usepackage{listings}
\usepackage{physics}
\usepackage{varioref}

% varioref stuff from Anders
\labelformat{section}{section~#1}
\labelformat{subsection}{section~#1}
\labelformat{subsubsection}{paragraph~#1}
\labelformat{equation}{#1}
\labelformat{figure}{figure~#1}
\labelformat{table}{table~#1}



\begin{document}
\title{FYS3150 -- Project 3}
\author{
    \begin{tabular}{r l}
        Kristian Gregorius Hustad & (\texttt{krihus})\\
        Jonas Gahr Sturtzel Lunde & (\texttt{jonassl})
    \end{tabular}}
%\date{}    % if commented out, the date is set to the current date

\maketitle




% quote
\setlength{\epigraphwidth}{0.75\textwidth}
\renewcommand{\epigraphflush}{center}
\renewcommand{\beforeepigraphskip}{50pt}
\renewcommand{\afterepigraphskip}{100pt}
\renewcommand{\epigraphsize}{\normalsize}

\epigraph{Rockets are cool. There's no getting around that.}
{\textit{Elon Musk}}

% alternative quote
%\epigraph{The first principle is that you must not fool yourself -- and you are the easiest person to fool.}{\textit{Richard Feynman}}

\begin{abstract}
\noindent
In this report, ...

\textbf{Fill in abstract}
\end{abstract}

\vfill


\begin{center}
    GitHub repository at \url{https://github.com/KGHustad/FYS3150}
\end{center}

\newpage

%%% MACROS
\newcommand{\half}{\frac{1}{2}}
\newcommand{\dt}{{\Delta t}}
\newcommand{\dx}{{\Delta x}}
\newcommand{\bigO}{{\mathcal{O}}}

\newcommand{\arrnew}{{_{\mathrm{new}}}}



\section{Introduction}\label{sec:intro}
%\subsection*{Description of the nature of the problem}
\cite{mhj_lecture_notes} % must cite something to avoid compilation error when using bibtex

In this report, we aim to study the solar system and the motion of celestial bodies in a gravitational field.



\textbf{Fill in introduction}




\section{Discussion of methods}\label{sec:methods}
Newton's law of universal gravitation dictate that, for two celestial bodies, $\alpha$ and $\beta$, the force of attraction between the two is given by
\begin{equation}
F_G
=\frac{Gm_{\beta}m_{\alpha}}{r_{\alpha \leftrightarrow \beta}^2}
\label{eq:grav:newton}
\end{equation}

%For circular orbits, we additionally have
%\begin{equation}
%    F_G= \frac{m_{\alpha}v_{\alpha}^2}{r_{\alpha \leftrightarrow \beta}}
%\end{equation}

Here $v_{\alpha}$ is the velocity of $\alpha$ relative to the system's center of mass, which we will keep in origo, and $r_{\alpha \leftrightarrow \beta}$ is the distance between $\alpha$ and $\beta$
\begin{equation}
r_{\alpha \leftrightarrow \beta} = \norm{ \vec{x}_{\alpha} - \vec{x}_{\beta} }
\end{equation}

\eqref{eq:grav:newton} can be adjusted to account for Einstein's theory of relativity, yielding
\begin{equation}
F_G
=\frac{Gm_{\beta}m_{\alpha}}{r_{\alpha \leftrightarrow \beta}^2}
\left[1 + \frac{3l^2}{r^2c^2}\right]
\label{eq:grav:einstein}
\end{equation}

where $l = \norm{\vec{x} \times \vec{v}}$

We will study both the Newton's equation, \eqref{eq:grav:newton}, from classical mechanics and its relativistic revision, \eqref{eq:grav:einstein}. For the sake of simplicity, we will limit ourselves to solving these equations in two dimensions.



To determine the motion of a celestial body, $\alpha$, we need to solve the following two second order ordinary differential equations
\begin{align}
\frac{d^2x}{dt^2} &= \frac{F_x(x,y)}{m_{\alpha}} \label{eq:acc:x}\\
\frac{d^2y}{dt^2} &= \frac{F_y(x,y)}{m_{\alpha}} \label{eq:acc:y}
\end{align}
where $m_{\alpha}$ is the mass of $\alpha$.\\

We know the initial positions $\vec{x}$ and speeds $\vec{v}$ of all celestial bodies. By combining that information with \eqref{eq:acc:x} - \eqref{eq:acc:y}, we can solve the system, but first we need a numerical method!




\subsection{Taylor expansions}
We will use the following Taylor expansions of $x$ and $v$ around $t$ to derive our numerical methods
\begin{align}
    x(t + h) &= x(t) + x'(t) h + \frac{x''(t)}{2} h^2 + \bigO(h^3) \label{eq:taylor:pos1}\\
    &= x(t) + v(t) h + \frac{a(t)}{2} h^2 + \bigO(h^3) \label{eq:taylor:pos2} \\
    v(t + h) &= v(t) + v'(t) h + \frac{v''(t)}{2} h^2 + \bigO(h^3) \label{eq:taylor:vel1} \\
    &= v(t) + a(t) h + \frac{a'(t)}{2} h^2 + \bigO(h^3) \label{eq:taylor:vel2}
\end{align}

\subsection{Forward Euler}
Using the two leading terms of \eqref{eq:taylor:pos2} and \eqref{eq:taylor:vel2} and setting $\dt = h$, we obtain

\begin{align}
    \vec{x}(t + \dt) &= \vec{x}(t) + \vec{v}(t)\dt  \label{eq:forwardeuler:pos} \\
    \vec{v}(t + \dt) &= \vec{v}(t) + \vec{a}(t)\dt \label{eq:forwardeuler:vel}
\end{align}

\subsection{Velocity Verlet}

By using a forward difference, we can approximate $a'(t)$ as
\begin{align}
    a'(t) \approx \frac{a(t+h) - a(t)}{h} \label{eq:diffacc:approx}
\end{align}

Inserting \eqref{eq:diffacc:approx} into \eqref{eq:taylor:vel2} and introducing $\dt = h$, we arrive at

\begin{align}
\vec{x}(t + \dt) &= \vec{x}(t) + \vec{v}(t)\dt + \half \vec{a}(t) \dt^2 \label{eq:velverlet:pos} \\
\vec{v}(t + \dt) &= \vec{v}(t) + \frac{\vec{a}(t) + \vec{a}(t + \dt)}{2} \dt \label{eq:velverlet:vel}
\end{align}

\subsection{Adapting the methods to motion in a gravitational field}

We need to choose a way to approximate $a(t + \dt)$ in \eqref{eq:velverlet:vel}. Since we are dealing only with gravitational forces, the acceleration in the classical case depends only on the position, so we use \eqref{eq:velverlet:pos} to find $a(t + \dt)$, and then we can use \eqref{eq:velverlet:vel}.

With the relativistic case, however, the acceleration also depends on the velocity. We will just cheat a bit here and use $\vec{v}(t)$ instead of $\vec{v}(t + \dt)$.


\subsection{Discretization}
In astronomy, one usually deals with astronomical units, with the unit distance being the mean radius of Earth's orbit around the Sun and the unit time being a year. We will study the system for a period of $T$ years, and values of $t \in [0, T]$. We write $\vec{x}(i \dt)$ as $x_i$, where $x_i$ is a vector of length two, and likewise for velocity and acceleration.

We discretize the Forward Euler scheme, \eqref{eq:forwardeuler:pos} - \eqref{eq:forwardeuler:vel}, as

\begin{align}
    x_{i+1} &= x_{i} + v_{i}\dt \\
    v_{i+1} &= v_{i} + a_{i}\dt
\end{align}

and the Velocity Verlet scheme, \eqref{eq:velverlet:pos} - \eqref{eq:velverlet:vel} as

\begin{align}
    x_{i+1} &= x_{i} + v_{i}\dt + a_{i} \frac{\dt^2}{2} \\
    v_{i+1} &= v_{i} + \frac{a_{i} + a_{i+1}}{2}\dt
\end{align}


%We discretize $F_x(x(t),y(t))$ and $F_y(x(t),y(t))$ at n+1 equally spaced points $t_0, t_1,..., t_n$, such that $t_i - t_{i-1} = h$....blablabla\\

% \subsection{Arriving at integration methods}
% Arriving at the Forward Euler method
% \begin{align}
% x'(i+1) &= x'(i) + x''(x(t_i), y(t_i))h\\
% y'(i+1) &= y'(i) + y''(x(t_i), y(t_i))h\\
% x(i+1) &= x(i) + v_x(i)h\\
% y(i+1) &= y(i) + v_y(i)h
% \end{align}
% and the Velocity Verlet method
% \begin{align}
% x(i+1) &= x(i) + x'(i)h + 0.5x''(i)h^2\\
% y(i+1) &= y(i) + y'(i)h + 0.5y''(i)h^2\\
% x'(i+1) &= x'(i) + 0.5(x''(i) + x''(i+1))h\\
% y'(i+1) &= y'(i) + 0.5(y''(i) + y''(i+1))h
% \end{align}


\subsection{Algorithms}
\begin{algorithm}
\caption{Forward Euler} \label{alg:forward_euler}
\begin{algorithmic}[1]
  \Procedure{ForwardEuler}{$p, v, p\arrnew, v\arrnew, m, \dt, \mathrm{num\_bodies}$}



  \EndProcedure
\end{algorithmic}
\end{algorithm}





\section{Implementation and results}\label{sec:implementation_and_results}
\subsection{Program structure}
Since we are of the opinion that an object oriented program structure generally is poorly suited for scientific programming, we perform all of our computations on arrays, without the notion of objects.

Operating on arrays is also optimal for accessing memory efficiently. Our program has two backends for carrying out the computations. One is done in Python with use of numpy's array arithmetic where possible, and the other is done in C.

We have, however, created a \textbf{make terrible OO Python program}
\\
\textbf{we can mention our ajust-sun function and our constant origo center of mass (which we have a print that shows doesn't move from there). idk where it fits tho.}
\\
\subsection{Unit Tests}
\subsubsection{Integration Stability}
We see that with the great Velocity Verlet integration method, the orbits quickly converge into a correct eliptical shape at timesteps as large as $\frac{1}{100}$ of a year, while the Forward Euler algorithm is just terrible. Our code uses Velocity Verlet, and timesteps several orders of magnitude smaller than used in this test, so we should expect rather accurate results.\\
\textbf{dt test plot thingy}
\subsubsection{Escape Velocity}
The analytical escape velocity at a distance $r$ from an object with mass $M$ is given as $\sqrt{\frac{2GM}{r}}$. Inserting the mass of the sun(1 Solar Mass), and the Earth-Sun distance(1 AU), we arrive at the escape velocity
\begin{equation}
v_{esc} = \sqrt{\frac{2\cdot 4\pi^2 \cdot 1}{1}} = \sqrt{8}\pi = 2.8284\pi AU/year
\end{equation}
\bftext{Insert escape velocity graphic and talk about how it matches analytical results.}
\subsubsection{Conservation of Energy and Angular Momentum}
Because we have an isolated system without external forces, we expect the total mechanical energy and angular momentum of the system to be conserved. In addition, if the planets orbit is circular, we expect the kinetic and potential energy to be respectively conserved. This is because the gravitational force from the sun always stays orthogonal on the planets velocity, meaning no work is done on the planet, and potential and kinetic energy stays unchanged.\\
\textbf{insert the two plots showing energy in the eliptical and spherical orbits(velocities are 2pi and 2.5pi. should prolly say that somewhere)}\\
\textbf{we have a print showing the low relative error in angular momentum. we also have a plot, but it's kinda boring, but use it if you want. it shows the angular momentum is conserved.}

\section{Conclusion}\label{sec:conclusion}
We have shown the extreme precision of the Velocity Verlet algorithm in regard to position-dependent forces, far outranking the Forward Euler method.
We then observed that the precession of Mercurys perhelion can only be accurately explained with the introduction of general relativity.
We have observed the large stability of ordinary planitary orbits, and increadible unstability of a planitary system with several massive objects, like our Sun-Jupiter system.

%\bibliographystyle{plain}
%\bibliographystyle{siam}
\bibliographystyle{IEEEtran}
\bibliography{../papers}{}

\end{document}
\grid
\grid
