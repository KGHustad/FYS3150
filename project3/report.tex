\documentclass[a4paper]{article}

\usepackage[utf8]{inputenc}
\usepackage[T1]{fontenc,url}
\usepackage{cite}
\usepackage{hyperref}
\usepackage{amsmath, amssymb}
\usepackage{tikz}
\usepackage{graphicx}
\usepackage{subcaption}
\usepackage{parskip}
\usepackage{lmodern}
\usepackage{algorithm}
\usepackage{algpseudocode}
\usepackage{epigraph}
\usepackage{listings}
\usepackage{todonotes}
\usepackage{physics}


\begin{document}
\title{FYS3150 -- Project 3}
\author{
    \begin{tabular}{r l}
        Kristian Gregorius Hustad & (\texttt{krihus})\\
        Jonas Gahr Sturtzel Lunde & (\texttt{jonassl})
    \end{tabular}}
%\date{}    % if commented out, the date is set to the current date

\maketitle



% quote
\setlength{\epigraphwidth}{0.75\textwidth}
\renewcommand{\epigraphflush}{center}
\renewcommand{\beforeepigraphskip}{50pt}
\renewcommand{\afterepigraphskip}{100pt}
\renewcommand{\epigraphsize}{\normalsize}

\epigraph{Rockets are cool. There's no getting around that.}
{\textit{Elon Musk}}

% alternative quote
%\epigraph{The first principle is that you must not fool yourself -- and you are the easiest person to fool.}{\textit{Richard Feynman}}

\begin{abstract}
\noindent
In this report, ...
\todo{Fill in abstract}
\end{abstract}

\vfill


\begin{center}
    GitHub repository at \url{https://github.com/KGHustad/FYS3150}
\end{center}

\newpage

%%% MACROS
\newcommand{\half}{\frac{1}{2}}
\newcommand{\dt}{{\Delta t}}
\newcommand{\dx}{{\Delta x}}
\newcommand{\bigO}{{\mathcal{O}}}



\section{Introduction}\label{sec:intro}
%\subsection*{Description of the nature of the problem}
\cite{mhj_lecture_notes} % must cite something to avoid compilation error when using bibtex

In this report, we aim to study the solar system and show \todo{Fill in introduction}




\section{Discussion of methods}\label{sec:methods}
The laws of motion dictate that, for two celestial bodies, $\alpha$ and $\beta$, the force of attraction between the two is given by
\begin{equation}
F_G= \frac{m_{\alpha}v_{\alpha}^2}{r_{\alpha \leftrightarrow \beta}}
=\frac{Gm_{\beta}m_{\alpha}}{r_{\alpha \leftrightarrow \beta}^2},
\end{equation}

Here $v_{\alpha}$ is the velocity of $\alpha$ relative to the system's center of mass, which we will keep in origo, and $r_{\alpha \leftrightarrow \beta}$ is the distance between $\alpha$ and $\beta$
\begin{equation}
r_{\alpha \leftrightarrow \beta} = \norm{ \vec{x}_{\alpha} - \vec{x}_{\beta} }
\end{equation}

\subsection{Forward Euler}
\subsection{Velocity Verlet}

We can derive the Velocity Verlet method through a Taylor expansion of $x$ and $v$ around $t$
\begin{align}
    x(t + h) &= x(t) + x'(t) h + \frac{x''(t)}{2} h^2 + O(h^3)\\
    &= x(t) + v(t) h + \frac{a(t)}{2} h^2 + O(h^3) \\
    v(t + h) &= v(t) + v'(t) h + \frac{v''(t)}{2} h^2 + O(h^3)\\
    &= v(t) + a(t) h + \frac{a'(t)}{2} h^2 + O(h^3) \label{eq:verlet:taylor:vel2}
\end{align}

By using a forward difference, we can approximate $a'(t)$ as
\begin{align}
    a'(t) \approx \frac{a(t+h) - a(t)}{h} \label{eq:diffacc:approx}
\end{align}

Inserting \eqref{eq:diffacc:approx} into \eqref{eq:verlet:taylor:vel2} and introducing $\dt = h$, we arrive at

%\todo{Need derivation of method}

\begin{align}
\vec{x}(t + \dt) &= \vec{x}(t) + \vec{v}(t)\dt + \half \vec{a}(t) \dt^2 \label{eq:velverlet:pos} \\
\vec{v}(t + \dt) &= \vec{v}(t) + \frac{\vec{a}(t) + \vec{a}(t + \dt)}{2} \label{eq:velverlet:vel} \dt
\end{align}

We need to choose a way to approximate $a(t + \dt)$. Since we are dealing only with gravitational forces, the acceleration depends only on the position, so we use \eqref{eq:velverlet:pos} to find $a(t + \dt)$, and then we can use \eqref{eq:velverlet:vel}.



We have the two second order ordinary differential equations
\begin{align}
\frac{d^2x}{dt^2} &= \frac{F_x(x,y)}{M_E} \\
\frac{d^2y}{dt^2} &= \frac{F_y(x,y)}{M_E}
\end{align}
where $M_E$ is the earth's mass.\\

\subsection{Discretization}
We discretize $F_x(x(t),y(t))$ and $F_y(x(t),y(t))$ at n+1 equally spaced points $t_0, t_1,..., t_n$, such that $t_i - t_{i-1} = h$....blablabla\\
\\
\subsection{Arriving at integration methods}
Arriving at the Forward Euler method
\begin{align}
x'(i+1) &= x'(i) + x''(x(t_i), y(t_i))h\\
y'(i+1) &= y'(i) + y''(x(t_i), y(t_i))h\\
x(i+1) &= x(i) + v_x(i)h\\
y(i+1) &= y(i) + v_y(i)h
\end{align}
and the Velocity Verlet method
\begin{align}
x(i+1) &= x(i) + x'(i)h + 0.5x''(i)h^2\\
y(i+1) &= y(i) + y'(i)h + 0.5y''(i)h^2\\
x'(i+1) &= x'(i) + 0.5(x''(i) + x''(i+1))h\\
y'(i+1) &= y'(i) + 0.5(y''(i) + y''(i+1))h
\end{align}


\section{Implementation and results}\label{sec:implementation_and_results}



\section{Conclusion}\label{sec:conclusion}


%\bibliographystyle{plain}
%\bibliographystyle{siam}
\bibliographystyle{IEEEtran}
\bibliography{../papers}{}

\end{document}
\grid
\grid
