\documentclass[a4paper]{article}

\usepackage[utf8]{inputenc}
\usepackage[T1]{fontenc,url}
\usepackage{cite}
\usepackage{hyperref}
\usepackage{amsmath, amssymb}
\usepackage{tikz}
\usepackage{graphicx}
\usepackage{parskip}
\usepackage{lmodern}
\usepackage{algorithm}
\usepackage{algpseudocode}
\usepackage{epigraph}
\usepackage{listings}


\begin{document}
\title{FYS3150 -- Project 2}
\author{
    \begin{tabular}{r l}
        Kristian Gregorius Hustad & (\texttt{krihus})\\
        Jonas Gahr Sturtzel Lunde & (\texttt{jonassl})
    \end{tabular}}
%\date{}    % if commented out, the date is set to the current date

\maketitle



% quote
\setlength{\epigraphwidth}{0.75\textwidth}
\renewcommand{\epigraphflush}{center}
\renewcommand{\beforeepigraphskip}{50pt}
\renewcommand{\afterepigraphskip}{100pt}
\renewcommand{\epigraphsize}{\normalsize}

\epigraph{Should array indices start at 0 or 1?  My compromise of 0.5 was rejected without, I thought, proper consideration.}
{\textit{Stan Kelly-Bootle}}

% alternative quote
%\epigraph{The first principle is that you must not fool yourself -- and you are the easiest person to fool.}{\textit{Richard Feynman}}

\begin{abstract}
\noindent
In this report, we show how the probability-distribution of two electrons in an harmonic oscillator well can be solved as an eigenvalue-problem. We will be looking at the behaviour of the probability, with, and without the implementation of the colomb-force, and different oscilator potential, solved with Jacobi's method.
\end{abstract}

\vfill


\begin{center}
    GitHub repository at \url{https://github.com/KGHustad/FYS3150}
\end{center}

\newpage

%%% MACROS
\newcommand{\half}{\frac{1}{2}}
\newcommand{\dx}{{\Delta x}}
\newcommand{\bigO}{{\mathcal{O}}}



\section{Introduction}\label{sec:intro}
%\subsection*{Description of the nature of the problem}
Schroedingers equation for an electron in a three-dimentional harmonic oscillator well in spherical coordinates, ignoring the colomb force:
\begin{equation}
  -\frac{\hbar^2}{2 m} \frac{d^2}{dr^2} u(r) 
       + \left ( V(r) + \frac{l (l + 1)}{r^2}\frac{\hbar^2}{2 m}
                                    \right ) u(r)  = E u(r) .
\end{equation}
If we discretize this equation at the groudstate of the orbital momentum of the electrons, it can be rewritten as an eigenvalue-problem with a tri-diagonal matrix:
\begin{equation*}
A\textbf{x} = \lambda \textbf{x}
\end{equation*}
where
\begin{equation*}
A = \begin{bmatrix} \frac{2}{h^2}+V_1 & -\frac{1}{h^2} & 0   & 0    & \dots  &0     & 0 \\
                                -\frac{1}{h^2} & \frac{2}{h^2}+V_2 & -\frac{1}{h^2} & 0    & \dots  &0     &0 \\
                                0   & -\frac{1}{h^2} & \frac{2}{h^2}+V_3 & -\frac{1}{h^2}  &0       &\dots & 0\\
                                \dots  & \dots & \dots & \dots  &\dots      &\dots & \dots\\
                                0   & \dots & \dots & \dots  &-\frac{1}{h^2}  &\frac{2}{h^2}+V_{N-2} & -\frac{1}{h^2}\\
                                0   & \dots & \dots & \dots  &\dots       &-\frac{1}{h^2} & \frac{2}{h^2}+V_{N-1}
             \end{bmatrix}
\end{equation*}
It has been shown \cite{mhj_lecture_notes} that...

\section{Discussion of methods}\label{sec:methods}
We can solve this eigenvalue-problem by a series of rotations that preserve the orthogonality of the matrix, through Jacobi's rotational method, described in \cite{mhj_lecture_notes}.
\subsection{Preservation of orthogonality}

The transformation $\textbf{w}_i = \textbf{U}\textbf{v}_i$, where U is an orthogonal matrix, is defined as a unitary transformation.
\\
We will show that the unitary transformation $\textbf{w}_i = \textbf{U}\textbf{v}_i$ preserves the the orthogonality of the vector, which can be tested with the dot product $\textbf{v}_j^T\textbf{v}_i = \delta_{ij}$. The dot product of the vectors after the unitary transformation, $\textbf{w}_j^T\textbf{w}_i$, should stay the same.\\
\\
\begin{equation}
\textbf{w}_j^T\textbf{w}_i = (\textbf{U}\textbf{v}_i)^T(\textbf{U}\textbf{v}_j) = (\textbf{v}_i^T\textbf{U}^T)(\textbf{U}\textbf{v}_j) = \textbf{v}_i^T\textbf{I}\textbf{v}_j = \textbf{v}_i^T\textbf{v}_j = \delta_{ij}
\end{equation}
\\
Notes;\\
$(\textbf{U}\textbf{v}_i)^T = \textbf{v}_i^T\textbf{U}^T$,\\
$\textbf{U}^T\textbf{U} = \textbf{I}$ if $\textbf{U}$ is orthogonal.

\section{Implementation and results}\label{sec:implementation_and_results}
\subsection{Implementing unit tests}

\section{Conclusion}\label{sec:conclusion}
We see that decreasing the strength of the ocillator potential $\omega$ pushes the probability-distribution further away from 0, making it shallower. The same goes for the introduction of the colomb force.


%\bibliographystyle{plain}
%\bibliographystyle{siam}
\bibliographystyle{IEEEtran}
\bibliography{../papers}{}

\end{document}
