\documentclass[a4paper]{article}

\usepackage[utf8]{inputenc}
\usepackage[T1]{fontenc,url}
\usepackage{cite}
\usepackage{hyperref}
\usepackage{amsmath, amssymb}
\usepackage{tikz}
\usepackage{graphicx}
\usepackage{parskip}
\usepackage{lmodern}
\usepackage{algorithm}
\usepackage{algpseudocode}
\usepackage{epigraph}
\usepackage{listings}


\begin{document}
\title{FYS3150 -- Project 2}
\author{
    \begin{tabular}{r l}
        Kristian Gregorius Hustad & (\texttt{krihus})\\
        Jonas Gahr Sturtzel Lunde & (\texttt{jonassl})
    \end{tabular}}
%\date{}    % if commented out, the date is set to the current date

\maketitle



% quote
\setlength{\epigraphwidth}{0.75\textwidth}
\renewcommand{\epigraphflush}{center}
\renewcommand{\beforeepigraphskip}{50pt}
\renewcommand{\afterepigraphskip}{100pt}
\renewcommand{\epigraphsize}{\normalsize}

\epigraph{Should array indices start at 0 or 1?  My compromise of 0.5 was rejected without, I thought, proper consideration.}
{\textit{Stan Kelly-Bootle}}

% alternative quote
%\epigraph{The first principle is that you must not fool yourself -- and you are the easiest person to fool.}{\textit{Richard Feynman}}

\begin{abstract}
\noindent
In this report, we show ...
\end{abstract}

\vfill


\begin{center}
    GitHub repository at \url{https://github.com/KGHustad/FYS3150}
\end{center}

\newpage

%%% MACROS
\newcommand{\half}{\frac{1}{2}}
\newcommand{\dx}{{\Delta x}}
\newcommand{\bigO}{{\mathcal{O}}}



\section{Introduction}\label{sec:intro}
%\subsection*{Description of the nature of the problem}

It has be shown \cite{mhj_lecture_notes} that ...

\section{Discussion of methods}\label{sec:methods}


\section{Implementation and results}\label{sec:implementation_and_results}

\section{Conclusion}\label{sec:conclusion}


%\bibliographystyle{plain}
%\bibliographystyle{siam}
\bibliographystyle{IEEEtran}
\bibliography{../papers}{}

\end{document}
