\documentclass[12pt,a4paper]{article}
\usepackage{amsmath}
\begin{document}

\section{exc a)}

Def \textit{unitary transformation}; $\textbf{w}_i = \textbf{U}\textbf{v}_i$, where U is an orthogonal matrix.\\
\\
We will show that the unitary transformation $\textbf{w}_i = \textbf{U}\textbf{v}_i$ preserves the dot product $\textbf{v}_j^T\textbf{v}_i = \delta_{ij}$, meaning that the dot product of the vectors after the unitary transformation stays the same. $\textbf{w}_j^T\textbf{w}_i = \delta_{ij}$.\\
\\
$\textbf{w}_j^T\textbf{w}_i = (\textbf{U}\textbf{v}_i)^T(\textbf{U}\textbf{v}_j) = (\textbf{v}_i^T\textbf{U}^T)(\textbf{U}\textbf{v}_j) = \textbf{v}_i^T\textbf{I}\textbf{v}_j = \textbf{v}_i^T\textbf{v}_j = \delta_{ij}$\\
\\
Notes;\\
$(\textbf{U}\textbf{v}_i)^T = \textbf{v}_i^T\textbf{U}^T$,\\
$\textbf{U}^T\textbf{U} = \textbf{I}$ if $\textbf{U}$ is orthogonal.

\section{exc b}
We need a solver for matrix (2).\\

We have a symetric orthogonal matrix A, and we're gonna reduce each element in the matrix to zero, by the following sequence:\\
1. Find the largest matrix element, $a_{kl}$.\\
2. Reduce the chosen element to zero by the following operations:\\
\\
(i) Calculate\\
\\
$\tau = \dfrac{a_{ll}-a_{kk}}{2a_{kl}}\\
\\
t = -\tau \pm \sqrt{1+\tau^2}\\
\\
c = \dfrac{1}{\sqrt{1+t^2}}, s = ct$.\\
\\
(ii) Calculate\\
\\
$b_{kk} = a_{kk}c^2 - 2a_{kl}cs + a_{ll}s^2\\
\\
b_{ll} = a_{ll}c^2 + a_{kl}cs + a_{kk}s^2$.\\
\\
Set the chosen element equal to zero.\\
\\
(iii) Loop over i in range(0,N+1), with $i \neq k,l$, and do the following:\\
\\
$b_{ik} = a_{ik}c - a_{il}s\\
\\
b_{il} = a_{il}c + a_{ik}s\\
\\
b_{ii} = a_{ii}\\
\\
$
The last one is obviously unnecessary if we overwrite A with B.\\
If overwriting, be careful not to use the overwritten values in the loop! (i.e. using the new $b_{ik}$ instead of the $a_{ik}$, which gets overwritten.)
\end{document}
