\documentclass[a4paper]{article}

\usepackage[utf8]{inputenc}
\usepackage[T1]{fontenc,url}
\usepackage{cite}
\usepackage{hyperref}
\usepackage{amsmath, amssymb}
\usepackage{tikz}
\usepackage{graphicx}
%\usepackage{subcaption}
\usepackage{parskip}
\usepackage{lmodern}
\usepackage{algorithm}
\usepackage{algpseudocode}
\usepackage{epigraph}
\usepackage{listings}
\usepackage{physics}
\usepackage{varioref}

% varioref stuff from Anders
\labelformat{section}{section~#1}
\labelformat{subsection}{section~#1}
\labelformat{subsubsection}{paragraph~#1}
\labelformat{equation}{(#1)}
\labelformat{figure}{figure~#1}
\labelformat{table}{table~#1}



\begin{document}
\title{FYS3150 -- Project 4}
\author{
    \begin{tabular}{r l}
        Kristian Gregorius Hustad & (\texttt{krihus})
    \end{tabular}}
%\date{}    % if commented out, the date is set to the current date

\maketitle




% quote
\setlength{\epigraphwidth}{0.75\textwidth}
\renewcommand{\epigraphflush}{center}
\renewcommand{\beforeepigraphskip}{50pt}
\renewcommand{\afterepigraphskip}{100pt}
\renewcommand{\epigraphsize}{\normalsize}

\epigraph{Nobody actually creates perfect code the first time around, except me.}
{\textit{Linus Torvalds}}

% alternative quote
%\epigraph{The first principle is that you must not fool yourself -- and you are the easiest person to fool.}{\textit{Richard Feynman}}

\begin{abstract}
\noindent
In this report, ...
\end{abstract}

\vfill


\begin{center}
    GitHub repository at \url{https://github.com/KGHustad/FYS3150}
\end{center}

\newpage

%%% MACROS
\newcommand{\half}{\frac{1}{2}}
\newcommand{\dt}{{\Delta t}}
\newcommand{\dx}{{\Delta x}}
\newcommand{\bigO}{{\mathcal{O}}}

\newcommand{\supnew}{^{\mathrm{new}}}



\section{Introduction}\label{sec:intro}
%\subsection*{Description of the nature of the problem}
\cite{mhj_lecture_notes} % must cite something to avoid compilation error when using bibtex

We aim to study ...


Methods are derived in \ref{sec:methods}, implementation considerations and results are given in \ref{sec:implementation_and_results}, and finally conclusions are drawn in \ref{sec:conclusion}.



\section{Discussion of methods}\label{sec:methods}



\section{Implementation and results}\label{sec:implementation_and_results}
For our implementation, we chose a hybrid Python-C approach -- inexpensive operations such as initialization of arrays and extracting the important quantities from the results are done in Python, where we can write short, high-level code, while the expensive operations, i.e. the Metropolis algorithm, is carried out in fast C code.



\section{Conclusion}\label{sec:conclusion}

%\bibliographystyle{plain}
%\bibliographystyle{siam}
\bibliographystyle{IEEEtran}
\bibliography{../../papers}{}

\end{document}
