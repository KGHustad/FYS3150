\documentclass[10pt,a4paper]{article}

%__Fonts and layout__
\usepackage[utf8]{inputenc}	%Allows input of unusual characters
\usepackage[T1]{fontenc,url}	%Helps correctly display unusual characters
\usepackage{parskip}	%Something about paragraph spacing
\usepackage{lmodern}	%Makes your font prettier or something
\usepackage{microtype}	%Makes text look prettier
\usepackage{verbatim}

%__math__
\usepackage{amsmath, amssymb}	%Improved math syntax and symbols
\usepackage{tikz}	%Graph drawing
\usepackage{physics}	%Mathematical physics notation
\usepackage{mathtools}	%Matrixes and shit

%__algorithms and programming__
\usepackage{algorithm}	%Allows writing of pretty algorithm. Perfect for desplaying code-syntax
\usepackage{algpseudocode}	%Similar to algorithm, just different layout
\usepackage{listings}	%Listing-enviroment, perfect for code or terminal output
\usepackage{enumerate}	%For listing of stuff

%__graphs and pictures__
\usepackage{graphicx}	%Includegraphics
\usepackage{float}	%Allows the [H] option, to force graphics in place

%__quotes and refrencing__
\usepackage{epigraph}	%Allows epigraph enviroment, for quotes
\usepackage{hyperref}	%Allows hyperreferences in pdf
\usepackage[backend=biber]{biblatex}	%Citations
\addbibresource{cites.bib}	%Reference to citation file


\begin{document}



%__Making a first-page__
\title{FYS3150 - Project 4}
\author{
	\begin{tabular}{rl}
		Jonas Gahr Sturtzel Lunde - (\textit{jonassl})\\
	\end{tabular}}
\date{10.11.2016}
\maketitle



%__Epigraph__
\setlength{\epigraphwidth}{0.75\textwidth}
\renewcommand{\epigraphflush}{center}
\renewcommand{\beforeepigraphskip}{50pt}
\renewcommand{\afterepigraphskip}{100pt}
\renewcommand{\epigraphsize}{\normalsize}
\epigraph{This is a quote}
	{\textit{By this guy}}



%__Abstract__
\begin{abstract}
\noindent
This is an abstract
\end{abstract}
\pagebreak



This is a cite\cite{lecture_notes}

\[
\begin{matrix*}
\uparrow & \downarrow \\
\downarrow & \uparrow
\end{matrix*}
\]

\[
\begin{matrix*}[r]
\uparrow & \downarrow \\
\downarrow & \uparrow
\end{matrix*}
\]


The partition function is given as
\begin{equation}
Z = \sum\limits_{i=1}^Q e^{\beta E_i}
\end{equation}

where the energy $E_i$ of a state $i$ is given as, and Q is the total possible number of states, $2^N$
\begin{equation}
E_i = -J \sum\limits_{<kl>}^N s_k s_l
\end{equation}

For our case we have a total of $Q = 16$ possible states. Of these states,
\begin{itemize}
\item 12 have an energy $E_i = 0$
\item 2 have an energy $E_i = -8J$
\item 2 have an energy $E_i = 8J$.
\end{itemize}
This gives a partition function
\begin{equation}
Z = \sum\limits_{i=1}^16 e^{\beta E_i} = 2e^{-8\beta J} + 2e^{8\beta J} + 12
\end{equation}

The expectation value for the energy $E$ is given by
\begin{equation}\begin{split}
\langle E \rangle &= -\frac{1}{Z}\frac{\partial Z}{\partial\beta} = -\frac{1}{Z} \frac{\partial (2e^{-8\beta J} + 2e^{8\beta J} + 12)}{\partial \beta}\\\\
&= -\frac{-16Je^{-8\beta J} + 16Je^{8\beta J}}{Z} = -\frac{-8Je^{-8\beta J} + 8Je^{8\beta J}}{e^{-8\beta J} + e^{8\beta J} + 6}
\end{split}\end{equation}

The expectation value for the squared energy $E^$ is given by
\begin{equation}\begin{split}
\langle E^2 \rangle &= \frac{1}{Z}\frac{\partial^2 Z}{\partial\beta^2} = \frac{1}{Z}\frac{\partial}{\partial\beta}\left(-16Je^{-8\beta J} + 16Je^{8\beta J} \right)\\\\
&= \frac{128J^2e^{-8\beta J} + 128J^2e^{8\beta J}}{Z} = \frac{64J^2e^{-8\beta J} + 64J^2e^{8\beta J}}{e^{-8\beta J}+e^{8\beta J} + 6}
\end{split}\end{equation}


The heat capacity $C_V$ is given by
\begin{equation}
\langle C_v \rangle = \frac{\langle E^2 \rangle - \langle E \rangle ^2}{k_B T^2} = \frac{\frac{\left(-8Je^{-8\beta J} + 8Je^{8\beta J}\right)^2}{\left(e^{-8\beta J} + e^{8\beta J} + 6\right)^2} - \frac{64J^2e^{-8\beta J} + 64J^2e^{8\beta J}}{e^{-8\beta J}+e^{8\beta J} + 6}}{k_B T^2}
\end{equation}


\begin{equation}
\langle |M| \rangle = \frac{1}{Z} \sum\limits_{i=1}^Q |M_i| e^{-\beta E_i}
\end{equation}
Of the total possible states $Q = 16$, we have
\begin{itemize}
\item 8 states with absolute magnetizaton $|M| = 2$ and energy $E = 0$
\item 2 states with absolute magnetization $|M| = 4$ and energy $E = -8J$
\item 6 states with no magnetization.
\end{itemize}
This results in a mean magnetization
\begin{equation}
\langle |M| \rangle = \frac{1}{2e^{-8\beta J} + 2e^{8\beta J} + 12} (16+8e^{8\beta}) = \frac{8+4e^{-8\beta J}}{e^{-8\beta J} + e^{\beta} + 6}
\end{equation}

\begin{equation}
\langle M \rangle = \frac{1}{2e^{-8\beta J} + 2e^{8\beta J} + 12} (0) = 0
\end{equation}

\begin{equation}
\langle M^2 \rangle = \frac{1}{2e^{-8\beta J} + 2e^{8\beta J} + 12} (128+64e^{8\beta}) = \frac{64 + 32e^{8\beta}}{e^{-8\beta J} + e^{8\beta J} + 6}
\end{equation}









\printbibliography %Prints citations. Should be at the bottom of document

\end{document}
