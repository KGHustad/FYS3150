\subsection{Parallelization of the 2D explicit scheme}
\subsubsection{Work distribution and granularity}
\cite{inf3380_bok}
With the explicit scheme, we can compute $v_{ij}^l$ for all $i, j \in \indexsetinner$, if all values at the previous time step are known. With Dirichlet boundary conditions, no action is required -- the boundary remains constant. Boundaries with Neumann conditions do need to be updated, and this must be done after the corresponding inner point has been computed.

Since there are $(N+2)^2$ independent values to be computed, the finest decomposition we could make, would be a 2D decomposition with $(N+2)^2$ workers. Such a decomposition would, however, be terribly inefficient due to the overhead incurred from the needed communication.

If one assumes that the values have already been distributed among the $\frac{(N+2)^2}{n^2}$ workers, and that each worker is responsible for computing $n^2$ values. This would lead to each worker having to communicate $4n$ values to its neighbour workers at each time step \footnote{The workers at the boundaries do not, of course, need to send the boundary points, but due to the implicit synchronization occuring when the workers wait for the required data to arrive, this does not affect the overall time used.}.
If we further assume that the workers are connected in a 2D mesh network, so that each worker is directly connected to its 4 neighbours, then we see that we have the communication at each time step runs in $\bigtheta (n)$ time.

In order to simplify matters, we use a 1D decomposition (along the y-axis).


\subsection{Unit tests}
We implemented a series of simple doctests for checking that all the algorithms work properly.
