\documentclass[10pt,a4paper]{article}

%__Fonts and layout__
\usepackage[utf8]{inputenc}	%Allows input of unusual characters
\usepackage[T1]{fontenc,url}	%Helps correctly display unusual characters
\usepackage{parskip}	%Something about paragraph spacing
\usepackage{lmodern}	%Makes your font prettier or something
\usepackage{microtype}	%Makes text look prettier
\usepackage{verbatim}

%__math__
\usepackage{amsmath, amssymb}	%Improved math syntax and symbols
\usepackage{mathtools}	%Some matrix stuff and shit
\usepackage{tikz}	%Graph drawing
\usepackage{physics}	%Mathematical physics notation

%__algorithms and programming__
\usepackage{algorithm}	%Allows writing of pretty algorithm. Perfect for desplaying code-syntax
\usepackage{algpseudocode}	%Similar to algorithm, just different layout
\usepackage{listings}	%Listing-enviroment, perfect for code or terminal output
\usepackage{enumerate}	%For listing of stuff

%__graphs and pictures__
\usepackage{graphicx}	%Includegraphics
\usepackage{float}	%Allows the [H] option, to force graphics in place

%__quotes and refrencing__
\usepackage{epigraph}	%Allows epigraph enviroment, for quotes
\usepackage{hyperref}	%Allows hyperreferences in pdf
%\usepackage[backend=biber]{biblatex}
%\addbibresource{cites.bib}

% gregorian macros
\newcommand{\indiceset}{\mathcal{I}}
\newcommand{\indicesetinner}{\mathcal{I}_{\mathrm{inner}}}


\begin{document}



%__Making a first-page__
\title{FYS3150 -- Project 5 -- PDE}
\author{
	\begin{tabular}{rl}
        Kristian Gregorius Hustad & (\texttt{krihus})\\
        Jonas Gahr Sturtzel Lunde & (\texttt{jonassl})
	\end{tabular}}
\date{December 9, 2016}
\maketitle



%__Epigraph__
\setlength{\epigraphwidth}{0.75\textwidth}
\renewcommand{\epigraphflush}{center}
\renewcommand{\beforeepigraphskip}{50pt}
\renewcommand{\afterepigraphskip}{100pt}
\renewcommand{\epigraphsize}{\normalsize}
\epigraph{This is a quote}
	{\textit{By this guy}}

\begin{abstract}
\noindent
This is an abstract
\end{abstract}

\pagebreak




\textbf{TODO}: Remember to cite \cite{hpl_fdm}!

\section{Method and Idea}
\subsection{One dimmentional case}
\subsubsection{Disrectization}
\begin{equation}
u(x+\Delta x, t+\Delta t) = u(x_{i+1},t_{i+1}) = u_{i+1,j+1}
\end{equation}

\subsubsection{Forward Euler - Explicit scheme}
\begin{align}
\frac{\partial u(x_i,t_j)^2}{\partial x^2} &= \frac{\partial u(x_i,t_j)}{\partial t} \\
\frac{u(x_i + \Delta x, t_j) - 2u(x_i,t_j) + u(x_i - \Delta x, t_j)}{\Delta x^2}
&= \frac{u(x_i, t_j + \Delta t) - u(x_i, t_i)}{\Delta t} \\
\frac{u_{i+1,j} - 2u_{i,j}+u_{i-1,j}}{\Delta x^2} &= \frac{u_{i,j+1} - u_{i,j}}{\Delta t} \\
u_{i,j+1} = \alpha u_{i-1,j} + (1-2\alpha)u_{i,j} + \alpha u_{i+1,j}
\end{align}
where $\alpha = \frac{\Delta t}{\Delta x^2}$


\subsubsection{Backward Euler - Implicit scheme}
\begin{align}
\frac{u_{i+1,j} - 2u_{i,j}+u_{i-1,j}}{\Delta x^2} &= \frac{u_{i,j} - u_{i,j-1}}{\Delta t} \\
u_{i,j-1} = \alpha u_{i+1,j} + (1 - 2\alpha )u_{i,j} + \alpha u_{i-1,j}
\end{align}

we define
\[
 = \begin{bmatrix}
1+2\alpha & -\alpha & 0 & \cdots & \cdots \\
-\alpha & 1+2\alpha & -\alpha & \cdots & \vdots \\
0 & \ddots & \ddots & \ddots & \vdots \\
\vdots & \ddots & \ddots & \ddots & -\alpha \\
\vdots & \cdots & \cdots & -\alpha & 1+2\alpha \\
\end{bmatrix}
\]
and
\[ V_j = \begin{bmatrix}
u_{0,j}\\
u_{1,j}\\
\cdots\\
u_{n-1,j}\\
u_{n,j}\\
\end{bmatrix}
\]
and get
\begin{equation}
ÂV_j = V_{j-1}
\end{equation}
which means we can define the state at time $j$, $V_j$ as
\begin{equation}
V_j = Â^{-j}V_0
\end{equation}


\subsubsection{The Crank-Nicolson scheme}


\subsubsection{The truncation error}


\subsection{Two dimmentional case}
\begin{equation}
\left(\frac{\partial^2 u(x,y,t)}{\partial x^2} + \frac{\partial^2 u(x,y,t)}{\partial y^2}\right) = \frac{\partial u(x,y,t)}{\partial t}
\end{equation}
Forward Euler
\begin{equation}
\frac{u_{i+1,j}^l - 2u_{i,j}^l + u_{i-1,j}^l}{h^2} + \frac{u_{i,j+1}^l - 2u_{i,j}^l + u_{i,j-1}^l}{h^2} = \frac{u_{i,j}^{l+1} - u_{i,j}^l}{\Delta t}
\end{equation}

\begin{equation}
u_{i,j}^{l+1} = u_{i,j}^l + \alpha\left( u_{i+1,j}^l + u_{i-1,j}^l + u_{i,j+1}^l + u_{i,j+1}^l - 4u_{i,j}^l \right)
\end{equation}



\section{Implementation and results}\label{sec:implementation_and_results}

\subsection{Parallelization of the 2D explicit scheme}
\subsubsection{Work distribution and granularity}
\cite{inf3380_bok}
We can compute $v_{ij}^l$ for all $i, j \in \indicesetinner$

In order to simplify matters, we use a 1D decomposition (along the y-axis).


% references
\bibliographystyle{IEEEtran}
\bibliography{../papers}{}

\end{document}
