\documentclass[10pt,a4paper]{article}

\usepackage[utf8]{inputenc}
\usepackage[T1]{fontenc,url}
\usepackage{parskip}
\usepackage{lmodern}
\usepackage{microtype}
\usepackage{verbatim}
\usepackage{amsmath, amssymb}
\usepackage{mathtools}
\usepackage{tikz}
\usepackage{physics}
\usepackage{algorithm}
\usepackage{algpseudocode}
\usepackage{listings}
\usepackage{enumerate}
\usepackage{graphicx}
\usepackage{float}
\usepackage{epigraph}
\usepackage{hyperref}
\usepackage[toc,page]{appendix}
\usepackage{varioref}
\usepackage{enumitem}
\usepackage{minted}

\definecolor{cbg_blue1}{rgb}{0.87843, 0.95686, 1.0}

\newenvironment{code_block}[1]{
\begin{minted}[bgcolor=cbg_blue1]{#1}}
{\end{minted}}


% varioref stuff from Anders
\labelformat{section}{section~#1}
\labelformat{subsection}{section~#1}
\labelformat{subsubsection}{paragraph~#1}
\labelformat{equation}{(#1)}
\labelformat{figure}{figure~#1}
\labelformat{table}{table~#1}


\newcommand{\program}[1]{\href{https://github.com/KGHustad/FYS3150/blob/master/project5/#1}{#1}}

\newcommand{\indexset}{\mathcal{I}}
\newcommand{\indexsetinner}{\mathcal{I}_{\mathrm{inner}}}
\newcommand{\bigO}{{\mathcal{O}}}
\newcommand{\bigtheta}{\Theta}
\newcommand{\half}{\frac{1}{2}}
\newcommand{\dt}{{\Delta t}}
\newcommand{\dx}{{\Delta x}}
\newcommand{\dy}{{\Delta y}}
\newcommand{\du}{{\Delta u}}
\newcommand{\fracpt}{\frac{\partial}{\partial t}}
\newcommand{\fracpx}{\frac{\partial}{\partial x}}
\newcommand{\fracpy}{\frac{\partial}{\partial y}}
\newcommand{\fracpxx}{\frac{\partial^2}{\partial x^2}}
\newcommand{\fracpyy}{\frac{\partial^2}{\partial y^2}}
\newcommand{\pt}{{\partial t}}
\newcommand{\px}{{\partial x}}
\newcommand{\py}{{\partial y}}
\newcommand{\pu}{{\partial u}}
\newcommand{\ppu}{{\partial^2 u}}
\newcommand{\pppu}{{\partial^3 u}}

\begin{document}



\title{FYS3150 -- Project 5 -- PDE}
\author{
	\begin{tabular}{rl}
        Kristian Gregorius Hustad & (\texttt{krihus})\\
        Jonas Gahr Sturtzel Lunde & (\texttt{jonassl})
	\end{tabular}}
\date{December 9, 2016}
\maketitle



\setlength{\epigraphwidth}{0.75\textwidth}
\renewcommand{\epigraphflush}{center}
\renewcommand{\beforeepigraphskip}{50pt}
\renewcommand{\afterepigraphskip}{100pt}
\renewcommand{\epigraphsize}{\normalsize}
\epigraph{Nobody reads my lecture notes}
	{\textit{Morten Hjorth-Jensen}}

\begin{abstract}
\noindent
This is an abstract
\end{abstract}

\vfill


\begin{center}
    GitHub repository at \url{https://github.com/KGHustad/FYS3150}
    or \url{https://github.uio.no/krihus/FYS3150} (UiO mirror)
\end{center}


\pagebreak

\tableofcontents



\section{Introduction}
In this report, we wish to study partial differential equations, a powerful tool for accurately estimating the characteristics of complex systems over time. Our main focus will be on numerical solutions, as most PDEs are too complicated for analytical solutions. The methods of choice are three different finite difference schemes - Forward Euler, Backward Euler and the Crank-Nicolson scheme.
\\\\
We have, however, in order to verify our results, chosen an analytically solvable PDE. The problem we will be employing, is a simplified case of the \textit{diffusion equation}. This specific case will be derived in \vref{sec:dif}, and is shown in one and two spatial dimensions below.

\begin{equation}\label{eqn:PDE}
\frac{\ppu(x,t)}{\px^2} = \frac{\pu(x,t)}{\pt}
\end{equation}

\begin{equation}\label{eqn:PDE2}
\frac{\ppu(x,y,t)}{\px^2} + \frac{\ppu(x,y,t)}{\py^2} = \frac{\pu(x,t)}{\pt}
\end{equation}


The diffusion equation is, naturally, used to model the process of diffusion, where large systems of randomly moving particles are described by it's macroscopic behaviour. The quantity $u(\textbf{r},t)$ may be physically interpreted as, among other things, the particle density, energy density, or temperature, at a given point in time and space.
\\\\
\textbf{TODO: Write some more about how we wish to implement the differential equation numerically}



We shall see how the tridiagonal solver described in \cite{hustad_lunde_project1} can be used to solve the backward Euler and Crank-Nicolson schemes efficiently in one dimension.

\section{Method and Idea}

\subsection{Studying the diffusion equation}\label{sec:dif}

The diffusion equation is normally defined as
\begin{equation}\label{eqn:dif}
\nabla \cdot \left[D(u,\textbf{r}) \ \nabla u(\textbf{r},t)\right] = \frac{\pu(\textbf{r},t)}{\pt}
\end{equation}
where $D(u,\textbf{r})$ is the diffusion coefficient and \textbf{r} is the position vector.
\\
Assuming the diffusion coefficient to be constant, the diffusion equation \vref{eqn:dif} collapses down to the heat equation:
\begin{equation}
D \nabla^2 u(\textbf{r},t) = \frac{\pu(\textbf{r},t)}{\pt}
\end{equation}
Since we are not looking at a specific physical interpretation of the diffusion equation, $D$ is left as an unknown constant. We will further simplify the equation by scaling our variables such that they become dimensionless, and $D$ disappears. This leaves us with the partial differential equation we will study:
\begin{equation}\label{eq:dif_simple}
\nabla^2 u(\textbf{r},t) = \frac{\pu(\textbf{r},t)}{\pt}
\end{equation}
Writing out the Laplace-operator, this becomes the equations \ref{eqn:PDE} and \ref{eqn:PDE2}, shown in the introduction, where $x$, $y$, and $t$ now are dimensionless variables.

\subsubsection{Source term}
It is possible to extend \vref{eq:dif_simple} with a source term, $f(x, y, t)$, however, it is not necessary for the computations we will be doing, and we will therefore not discuss source terms in this report.



\subsection{Discretization \& Notation}\label{sec:disc}
Each spatial dimension will be discretized as a total of $n+2$ points.\\
In one dimension:
\begin{equation} (x_i), \quad i \in [0,n+1] \end{equation}
and in two dimensions:
\begin{equation} (x_i, y_j), \quad i,j \in [0,n+1]\end{equation}

The time will be discretized as $\tau+1$ points.
\begin{equation} t_l, \quad l \in [0,\tau] \end{equation}
We will also rewrite this to the following compact notation from \cite{hpl_fdm} where spatial position is written in subscript and the temporal position in superscript \\
In one dimension:
\begin{equation}
u(x+\dx,\ t+\dt) = u(x_{i+1},\ t_{l+1}) = u_{i+1}^{l+1}
\end{equation}
and two dimensions:
\begin{equation}
u(x+\dx,\ y+\dy,\ t+\dt) = u(x_{i+1},\ y_{j+1},\ t_{l+1}) = u_{i+1,j+1}^{l+1}
\end{equation}


We will also write the derivatives in \ref{eqn:PDE} and \ref{eqn:PDE2} with the compact notations
\begin{equation} u_{xx} = u_t\end{equation}
\begin{equation} \ u_{xx} + u_{yy} = u_t \end{equation}


\subsection{Initial Conditions \& Boundary Conditions}
The system will be interpreted as a rod of length $L$ for the one-dimensional system, and a square with side-lengths $L$ for the two-dimensional system, such that $x,y \in [0,L]$. We will study the system over a time $T$. This gives a length
\begin{equation}
\dx = \dy = x_i - x_{i-1} = y_j - y_{j-1} = \frac{L}{n+1}
\end{equation}
between the spatial discretizations, and a time
\begin{equation}
\dt = t_l - t_{l-1} = \frac{T}{\tau}
\end{equation}
between the time discretizations.\\\\
The initial conditions of the system are set as
\begin{equation}
u(x,t) = 0, \quad 0 < x < L
\end{equation}
while the boundary conditions of our system are
\begin{equation}
u(0,t) = 0 \quad \quad u(L,t) = 1
\end{equation}




\subsection{One dimensional case}\label{sec:method:1d}
\subsubsection{The central difference scheme}\label{sec:cent}
The Forward and Backward Euler schemes will be relying on the three-point central difference scheme to approximate the spatial derivative. The scheme is derived from a Taylor expansion around $u(x,t)$, with time as variable, both forward and backwards.
\begin{equation}
u(x+\Delta x, t) = u(x,t) + \frac{\pu(x,t)}{\px}\dx + \frac{\ppu(x,t)}{2\px^2}\dx^2 + O(\dx^3)
\end{equation}

\begin{equation}
u(x-\Delta x, t) = u(x,t) - \frac{\pu(x,t)}{\px}\dx + \frac{\ppu(x,t)}{2\px^2}\dx^2 + O(\dx^3)
\end{equation}

adding both sides of the equations together, we get
\begin{equation}
u(x+\Delta x, t) + u(x-\Delta x, t) = 2u(x,t) + 2\frac{\ppu(x,t)}{2\px^2}\dx^2 + 2O(\dx^3)
\end{equation}

which, solving for the second derivative of $x$, gives
\begin{equation}\begin{split}
\frac{\ppu(x,t)}{2\px^2} &= \frac{u(x+\Delta x, t) - 2u(x,t) + u(x-\Delta x, t)}{\dx^2} + \frac{2O(\dx^3)}{\dx^2}\\
&\approx \frac{u(x+\Delta x, t) - 2u(x,t) + u(x-\Delta x, t)}{\dx^2}
\end{split}\end{equation}
Giving a truncation error running as $\dx$.\\\\
We can discretize the derivative as shown in \ref{sec:disc}.
\begin{equation}
u_{xx} = \frac{u_{i+1}^l - 2u_i^l+u_{i-1}^l}{\dx^2}
\end{equation}




\subsubsection{Forward Euler - Explicit scheme}
We will now derive the Explicit Forward Euler method, to approximate the partial differential equation \ref{eqn:PDE}. By Taylor expanding forward around $u(x,t)$, with time as variable, and truncating after the first derivative, we get
\begin{equation}
u(x,t+\dt) = u(x,t) + \frac{\pu(x,t)}{\pt}\dt + O(\dt^2)
\end{equation}
Solving for the time derivative gives
\begin{equation}
\frac{\pu(x,t)}{\pt} = \frac{u(x,t+\dt) - u(x,t)}{\dt} + \frac{O(\dt^2)}{\dt} \approx \frac{u(x,t+\dt) - u(x,t)}{\dt}
\end{equation}
which means we have a truncation error running as $\dt$.
\\\\
We can discretize this as shown in \ref{sec:disc}, giving
\begin{equation}
u_t = \frac{u_i^{l+1} - u_i^l}{\dt}
\end{equation}
Combining this with the central difference scheme, we get the approximation to the partial differential equation \vref{eqn:PDE}
\begin{equation}
\frac{u_{i+1}^l - 2u_i^l+u_{i-1}^l}{\dx^2} = \frac{u_i^{l+1} - u_i^l}{\dt}
\end{equation}

Solving for $u_i^{l+1}$, and introducing the known constant $\alpha = \frac{\dt}{\dx^2}$, we arrive at the explicit scheme
\begin{equation}\label{eq:1dscheme:fe}
u_i^{l+1} = \alpha u_{i-1}^l + (1-2\alpha)u_i^l + \alpha u_{i+1}^l
\end{equation}
We see that the state of the system at a time, $t_{l+1}$(left side), only depends on the conditions of the system in a previous state, $t_l$(right side), making this an explicit method.


\subsubsection{Backward Euler - Implicit scheme}\label{sec:method_be}
We will now derive the implicit Backward Euler method to approximate the time derivative of the partial differential equation. Taylor expanding backwards around $u(x,t)$, truncating after the first derivative, we get

\begin{equation}
u(x,t-\dt) = u(x,t) - \frac{\pu(x,t)}{\pt}\dt + O(\dt^2)
\end{equation}
Solving for the time derivative gives
\begin{equation}
\frac{\pu(x,t)}{\pt} = \frac{u(x,t) - u(x,t-\dt)}{\dt} + \frac{O(\dt^2)}{\dt} \approx \frac{u(x,t) - u(x,t-\dt)}{\dt}
\end{equation}
giving a truncation error running as $\dt$, just as the explicit scheme.
\\\\
We can discretize this as shown in \ref{sec:disc}, giving
\begin{equation}
u_t = \frac{u_i^l - u_i^{l-1}}{\dt}
\end{equation}
Again combining with the central difference scheme, we approximate the partial differential equation as
\begin{equation}
\frac{u_{i+1}^l - 2u_i^l+u_{i-1}^l}{\dx^2} = \frac{u_i^l - u_i^{l-1}}{\dt}
\end{equation}
Lastly, introducing $\alpha = \frac{\dt}{\dx^2}$, we can rewrite to
\begin{equation}\label{eq:1dscheme:be}
u_i^{l-1} = -\alpha u_{i+1}^l + (1 + 2\alpha )u_i^l - \alpha u_{i-1}^l
\end{equation}
\\
We see that we cannot write the state of the system as an explicit function of a previous state, making this an implicit scheme. If we write \ref{eq:1dscheme:be} out for all values of $i \in [0,n+1]$, we see that it becomes a set of linear equations, which we will write as the matrix equation
\begin{equation}\label{eq:be_matrix}
V_{l-1} = \hat{A}V_l
\end{equation}
where we define the three-diagonal matrix:
\begin{equation}
\hat{A} = \begin{bmatrix}
1+2\alpha & -\alpha & 0 & \cdots & \cdots \\
-\alpha & 1+2\alpha & -\alpha & \cdots & \vdots \\
0 & \ddots & \ddots & \ddots & \vdots \\
\vdots & \ddots & \ddots & \ddots & -\alpha \\
\vdots & \cdots & \cdots & -\alpha & 1+2\alpha \\
\end{bmatrix}
\end{equation}
and the state of the system at a given time $t_l$:
\begin{equation}\label{eqn:V} V_l = \begin{bmatrix}
u_{0}^l\\
u_{1}^l\\
\cdots\\
u_{n}^l\\
u_{n+1}^l\\
\end{bmatrix}
\end{equation}

This means we can define the state of the system, $V_l$, at time, $t_l$, as
\begin{equation} \label{eq:be_matrix_inverse}
V_l = \hat{A}^{-l}V_0
\end{equation}

While we could implement our algorithm using \ref{eq:be_matrix_inverse}, we will be using \ref{eq:be_matrix} so that we can employ our tridiagonal solver.

\subsubsection{The Crank-Nicolson scheme}
\textbf{TODO}: Deriving Crank-Nicolson scheme more thoroughly, among other tings: showing the truncation's error.

While the Forward and Backward Euler methods rely on taking a step forwards and backwards in time to approximate the derivative, the Crank-Nicolson scheme simply takes the average of the two. Moving the Backward Euler scheme one time step ahead, such that it's time derivative matches the Forward Euler's, we get the average:
\begin{equation}
\frac{u_i^{l+1}-u_i^l}{\dt} = \frac{1}{2}\left( \frac{u_{i+1}^l - 2u_i^l + u_{i-1}^l}{\dx^2} + \frac{u_{i+1}^{l+1} - 2u_i^{l+1} + u_{i-1}^{l+1}}{\dx^2}\right)
\end{equation}
After again defining $\alpha = \frac{\dt}{\dx^2}$, we can separate the two time steps $t_l$ and $t_{l-1}$, giving us the implicit Crank-Nicolson scheme.
\begin{equation}
-\alpha u_{i-1}^l + (2+2\alpha)u_i^l - \alpha u_{i+1}^l = \alpha u_{i-1}^{l-1} + (2-2\alpha)u_i^{l-1} + \alpha u_{i+1}^{l-1}
\end{equation}
As with the Backward Euler scheme, we write this as a system of linear equations for $i \in [0,n+1]$, which can be written as the matrix equation
\begin{equation}\label{eqn:crank}
(2\hat{I}+\alpha \hat{B})V_l = (2\hat{I}-\alpha \hat{B})V_{l-1}
\end{equation}
where $V_l$ is defined in \vref{eqn:V}. We also have the identity matrix $\hat{I}$, and the three-diagonal matrix:

\[
\hat{B} = \begin{bmatrix}
2 & -1 & 0 & \cdots & \cdots \\
-1 & 2 & -1 & 0 & \vdots \\
\vdots & \ddots & \ddots & \ddots & \vdots \\
\vdots & \ddots & \ddots & 2 & -1 \\
\vdots & \cdots & 0 & -1 & 2 \\
\end{bmatrix}
\]

We can rewrite \vref{eqn:crank} to define the state of the system $V_l$ as a function of it's previous state $V_{l-1}$:
\begin{equation}
V_l = (2\hat{I}+\alpha \hat{B})^{-1} (2\hat{I} - \alpha \hat{B})V_{l-1}
\end{equation}

Just as in \vref{sec:method_be} we will be using our tridiagonal solver instead of performing a matrix inversion. To see that this is indeed possible, observe that the only unknown in \vref{eqn:crank} is $V_l$, so it can be written as a matrix equation \[
A \mathbf{v} = \mathbf{s}
\] where $A = 2\hat{I}+\alpha \hat{B}$, $ \mathbf{s} = (2\hat{I} - \alpha \hat{B})V_{l-1}$ and $\mathbf{v} = V_l$.



\subsubsection{An analytical solution}\label{sec:analytical_1d}
\cite{inf-mat2351_book} gives a solution to the diffusion equation

\begin{align}\label{eq:analytical_1d_book}
    u(x, t) = e^{-\pi^2 t} \sin(\pi x) , \quad u(0, t) = u(1, t) = 0
\end{align}

We can easily show that \vref{eq:analytical_1d_book} solves \vref{eqn:PDE},
\begin{align}
\fracpt u(x, t) &= -\pi^2 u(x, t) \\
\fracpxx u(x, t) &= \fracpx \left(\pi e^{-\pi^2 t} \cos(\pi x) \right)
= -\pi^2 e^{-\pi^2 t} \sin(\pi x) = -\pi^2 u(x, t)
\end{align}



\subsection{Error and stability}
\label{sec:method:1d:error}
Since all methods build on the central difference scheme, the truncation error of the spatial derivative runs as $\dx^2$ for all schemes. From deriving the three methods in \ref{sec:method:1d} we know that the Euler methods has a truncation error running as $\dt$, while it is $\dt^2$ for the Crank-Nicolson scheme.
\\\\



\begin{table}[htb]
\begin{center}
\begin{tabular}{lcc}
 		& Truncation Error	& Stability
\\ \hline \\
Forward Euler	& $\dt$ and $\dx^2$	& only $\dt \le \frac{1}{2}\dx^2$ \\
\\ \hline \\
Backward Euler	& $\dt$ and $\dx^2$	& all $\dt$ and $\dx$ \\
\\ \hline \\
Crank-Nicolson	& $\dt^2$ and $\dx^2$	& all $\dt$ and $\dx$ \\
\\ \hline \\
\end{tabular}
\end{center}
\caption{Proportionality of truncation error and stability requirements for the three numerical schemes}
\label{table:error}
\end{table}


\subsection{Two dimensional case}
We will now expand our implementations to two spatial dimensions, resulting in the partial differential equation, $u_{xx} + u_{yy} = u_t$, or, written out:
\begin{equation}
\left(\frac{\ppu(x,y,t)}{\px^2} + \frac{\ppu(x,y,t)}{\py^2}\right) = \frac{\pu(x,y,t)}{\pt}
\label{eq:diffusion_2d}
\end{equation}
The central difference scheme derived for one dimension in \ref{sec:cent} can easily be expanded to two spatial dimensions, giving the left side of the PDE:
\begin{equation}
u_{xx} + u_{yy} = \frac{u_{i+1,j}^l - 2u_i^l+u_{i-1,j}^l}{\dx^2} + \frac{u_{i,j+1}^l - 2u_i^l+u_{i,j-1}^l}{\dy^2}
\end{equation}

As we have only introduced a new spatial dimension, the schemes for approximating time derivatives remains unchanged. We expand the notation of the Forward Euler method to two dimensions:
\begin{equation}
u_t = \frac{u_{i,j}^{l+1} - u_{i,j}^l}{\dt}
\end{equation}

Choosing $\dx = \dy = h$, and combining these two schemes, we can set up the two dimensional partial differential equation
\begin{equation}
\frac{u_{i+1,j}^l - 2u_{i,j}^l + u_{i-1,j}^l}{h^2} + \frac{u_{i,j+1}^l - 2u_{i,j}^l + u_{i,j-1}^l}{h^2} = \frac{u_{i,j}^{l+1} - u_{i,j}^l}{\dt}
\end{equation}

Solving for $u_{i,j}^{l+1}$ gives us the explicit scheme
\begin{equation}
u_{i,j}^{l+1} = u_{i,j}^l + \alpha\left( u_{i+1,j}^l + u_{i-1,j}^l + u_{i,j+1}^l + u_{i,j-1}^l - 4u_{i,j}^l \right)
\end{equation}
where $\alpha = \frac{\dt}{h^2}$.

\subsubsection{Extending the 1D case to two dimensions}
When extending
\begin{equation}
u(0,t) = 0 \quad \quad u(L,t) = 1
\end{equation}
to two dimensions, it is natural to let
\begin{equation}
\label{eq:2d:boundary}
u(0,y,t) = 0 \quad \quad u(L,y,t) = 1
\end{equation}

and
\begin{equation}
\label{eq:2d:initial}
u(x,y,0) = 0 \quad x < L
\end{equation}


The real question is how one should treat the boundaries in the $y$ dimension.
One choice could be
\begin{equation}
\fracpy u(x,0,t) = \fracpy u(x,L,t) = 0
\end{equation}

This would translate to having the boundaries perfectly insulated -- the flow of heat through the boundary is exactly zero. Combined with \ref{eq:2d:boundary} and \ref{eq:2d:initial}, this would imply that $u$ does not vary in the $y$ dimension, i.e.
\begin{equation}
u(x,y_i,t) = u(x,y_j,t) \quad i, j \in \indexset
\end{equation}

A more interesting choice might be
\begin{equation}
u(x,0,t)=u(x,L,t)=0 \quad x < L
\end{equation}
meaning that the right side is constantly 1 while all the other sides are zero. Here we would expect there to be variation in the $y$ dimension.

\subsubsection{An analytical solution}
We can adapt the analytical solution described in \vref{sec:analytical_1d} to solve the 2D problem.

Using the Dirichlet boundary conditions
\begin{align}
u(0, y, t) = u(1, y, t) = u(x, 0, t) = u(x, 1, t) = 0
\end{align}
we get
\begin{align}\label{eq:analytical_2d}
    u(x, y, t) = e^{-2\pi^2 t} \sin(\pi x) \sin(\pi y)
\end{align}

which solves \vref{eq:diffusion_2d} since
\begin{align}
\fracpt u(x, y, t) &= -2\pi^2 u(x, y, t) \\
\fracpxx u(x, y, t) &= \fracpx \left( \pi e^{-2\pi^2 t} \cos(\pi x) \sin(\pi y) \right) \\
&= -\pi^2 e^{-2\pi^2 t} \sin(\pi x) \sin(\pi y) \\
&= -\pi^2 u(x, y, t) \\
\fracpyy u(x, y, t) &= \fracpy \left( \pi e^{-2\pi^2 t} \sin(\pi x) \cos(\pi y) \right) \\
&= -\pi^2 e^{-2\pi^2 t} \sin(\pi x) \sin(\pi y) \\
&= -\pi^2 u(x, y, t)
\end{align}



\section{Implementation and results}\label{sec:implementation_and_results}
Note: $\dt$ is always set to $\frac{1}{2}\dx^2$ unless otherwise mentioned.\\\\

\subsection{1D -- Sinus case}
We will study how the various schemes by comparing them to the analytical solution derived in \ref{sec:analytical_1d}. However, before looking at the results, we should remind ourselves of how the methods differ and what we should expect.

First we should note that our analytical solution, \ref{eq:analytical_1d_book}, has the shape of a sine wave that decays over time. As time approaches infinity, i.e. $\lim_{t \to \infty} u(x, t) = 0$.

Forward Euler uses a forward difference in time, meaning it approximates the rate of change in the next time step from rate of change in the current time step. Since the analytical solution is decaying, this means that Forward Euler is at an advantage in terms of convergence rate.

Backward Euler, which uses a backward difference in time, makes the opposite assumption. The rate of change at the current time step is approximated from the rate of change at the next time step. We will see that this is disadvantageous.

Crank-Nicolson strikes a balance between Forward and Backward Euler. It does, however have a faster convergence rate in terms of $\dt$ as discussed in \ref{sec:method:1d:error}.

We clearly see the impact of the decreased $\dt$ value from 0.1 to 0.01.
\begin{figure}[H]
\includegraphics[width=\textwidth]{fig/{plot_T=0.10}.pdf}
\caption{Different PDE solvers with $\dx \in \{0.01,0.1\}$ plotted against analytical solution for $T=1$.}
\label{fig:}
\end{figure}

Same scenario as above, this time after ten times as long. Now the impact of the decreased $\dx$ is even clearer. We also see Forward Euler diverging the quickest, due to the nature of the problem. This is not always the case, as the Crank-Nicolson is usually more accurate.
\begin{figure}[H]
\includegraphics[width=\textwidth]{fig/{plot_T=1.00}.pdf}
\caption{Different PDE solvers with $\dx \in \{0.01,0.1\}$ plotted against analytical solution for $T=0.1$.}
\label{fig:}
\end{figure}

Here we see what we suspected from the last plot: the Forward Euler actually has the quickest convergence.
\begin{figure}[H]
\includegraphics[width=\textwidth]{fig/{error_T=1.00n=10}.pdf}
\caption{Absolute error of different PDE solvers for the $T=0.1$, $\dx=0.1$, $\dt=0.0005$ case.}
\label{fig:}
\end{figure}

\subsection{1D -- Linear case}
In this specific case we are not capable of distinguishing the different schemes on a plot, as they are behave much alike.
\begin{figure}[H]
\includegraphics[width=0.5\textwidth]{fig/{1D_linplot_T=0.10dx=0.10}.pdf}
\includegraphics[width=0.5\textwidth]{fig/{1D_linplot_T=0.10dx=0.01}.pdf}
\caption{Comparing $\dx = 0.1$ and $\dx = 0.01$ for the linear case.}
\label{fig:}
\end{figure}


Picking one of the methods, the Crank-Nicolson, we study how it diverges (we don't have an analytical solution here, but we know that it should converge on a linear solution).
\begin{figure}[H]
\includegraphics[width=\textwidth]{fig/{1D_linplot_Crank_nicolson}.pdf}
\caption{}
\label{fig:}
\end{figure}

\subsection{Implementing boundary conditions in two dimensions}
We generalized our code to allow for Dirichlet conditions, i.e. $u_{\mathrm{boundary}} = f(x, y)$ and Neumann conditions $\fracpx u_{\mathrm{x-boundary}} = 0$, $\fracpy u_{\mathrm{y-boundary}} = 0$. Each side can be set to either of these two boundary conditions.


\subsection{Tests}
We implemented a series of tests for checking that all the algorithms work properly. These are described more elaborately in \vref*{appx:testing}.


\bibliographystyle{IEEEtran}
\bibliography{../papers}{}

\newpage
\appendix
%\renewcommand{\thesection}{Appendix \Alph{section}}
\labelformat{section}{appendix~#1}
\section*{\Huge{Appendices}}

\section{Parallelization of the 2D explicit scheme}
\label{appx:parallel}

In this appendix, we will analyze the isoefficiency of the parallel algorithms we discuss as described in \cite{inf3380_bok}. We will let $N$ denote the number of spatial points in along either axis (recall that $\dx = \dy$) and $\tau+1$ the number of temporal points.

The time needed to perform the computation serially is thus given by
\begin{equation}
T_S = \bigtheta(\tau N^2)
\end{equation}

%But since there is a strict dependence between the currect time step and the previous one, we could just as well study the time needed to compute a single time step (with the single communication).

\subsection{Work distribution and granularity}
With the explicit scheme, we can compute $v_{ij}^l$ for all $i, j \in \indexsetinner$, if all values at the previous time step are known. With Dirichlet boundary conditions, no action is required -- the boundary remains constant. Boundaries with Neumann conditions do need to be updated, and this must be done after the corresponding inner point has been computed.


\subsubsection{2D decomposition}
\label{appx:parallel:analysis:2d}
Since there are $(N)^2$ independent values to be computed, the finest decomposition we could make, would be a 2D decomposition with $N^2$ workers. Such a decomposition would, however, be terribly inefficient due to the overhead incurred from the needed communication.

If one assumes that the values have already been distributed among the $\frac{N^2}{n^2}$ workers, and that each worker is responsible for computing $n^2$ values. This would lead to each worker having to communicate $4n$ values to its neighbour workers at each time step \footnote{The workers at the boundaries do not, of course, need to send the boundary points, but due to the implicit synchronization occuring when the workers wait for the required data to arrive, this does not affect the overall time used.}.
If we further assume that the workers are connected in a 2D mesh network, so that each worker is directly connected to its 4 neighbours, then we see that we have the communication at each time step runs in $\bigtheta (n)$ time. The work required to compute the $n^2$ values is obviously $\bigtheta(n^2)$.

This yields a parallel time of
\begin{equation}
T_P = \bigtheta(\tau(n^2 + n)) = \bigtheta(\tau n^2)
\end{equation}

This means that we get a speedup of
\begin{equation}
S = \frac{T_S}{T_P} = \frac{\bigtheta(\tau N^2)}{\bigtheta(\tau n^2)}
= \bigtheta\left( \frac{N^2}{n^2} \right)
\end{equation}

Which is asymptotically equal to the number of workers, and hence this decomposition is cost-optimal!

\subsubsection{1D decomposition}
\label{appx:parallel:analysis:1d}
While the 2D decomposition is nice for use with supercomputers, it does carry some limitations. Obviously the number of workers $\frac{N^2}{n^2}$ would need to be an integer, but since $\frac{N^2}{n^2} = \left(\frac{N}{n}\right)^2$ we see that in fact it need not only be an integer, but also a perfect square! \footnote{It is possible to modify this decomposition so that the number of workers is a product of two (not necessarily equal) integers, but the efficiency degrades as the workers grid of values deviates from a square.} This is really bad news for anyone wanting to run our program on their CPU with 8 logical processors.

In order to simplify matters, we use a 1D decomposition (along the y-axis). We let each of the $\frac{N}{n}$ workers be responsible for computing $n$ rows. Each worker would then at each time step have to communicate its entire upper and lower row to the workers above and below, respectively. This would take $\bigtheta(N)$ time. The computation at each time step would take $\bigtheta(nN)$ time.

We see that with this leads to a parallel time of
\begin{equation}
T_P = \bigtheta(\tau(nN + N)) = \bigtheta(\tau nN)
\end{equation}

and a speedup of
\begin{equation}
S = \frac{T_S}{T_P} = \frac{\bigtheta(\tau N^2)}{\bigtheta(\tau nN)}
= \bigtheta\left( \frac{N}{n} \right)
\end{equation}

which is asymptotically equal to the number of workers, and hence this decomposition is cost-optimal too.


\subsection{Implementation of the parallel algorithm}
Since the 2D decomposition discussed in \ref{appx:parallel:analysis:2d} involves an order of magnitude more hazzle, we chose to use the 1D decomposition discussed in \ref{appx:parallel:analysis:1d}.

We have made two implementation, one by extending our serial C code with OpenMP directives and one with explicit message passing with MPI.
Unfortunately, the part of the MPI code which carried out the data distribution was lost shortly after being written as one of authors' computer's motherboard fell into eternal sleep causing the encrypted data on local drive to be impossible to read.

\section{Asserting the correctness of the program}
\label{appx:testing}

Testing a bilingual software project such as ours is slightly more demanding than testing software written in a single language. The errors may lay in code written in either of the languages or even in the interfacing layer. We therefore conduct a variety of tests to ensure that all of our code is correct.

From Python, we assert that our algorithms are correctly implemented with standard unit tests as described in \ref{appx:testing:unit_tests}. Additionally, we have doctests which have the strength of being easy to read and also make it easy to see where the test failed and what the expected and computed values were.

In C, we check that there are no memory related errors by writing simple test programs that should provoke Valgrind to give an error if there are any memory leaks or uninitialised values. This is described in further detail in \ref{appx:testing:valgrind}.

Finally, as described in \ref{appx:testing:travis}, we have instructed Travis CI to run these tests at every push to our GitHub repository so that the test results can be viewed for each version, eliminating the need to run them manually when looking for the last working version for instance.

All tests can be run from the root directory of this project with:

\begin{minted}[bgcolor=cbg_blue1]{text}
$ make test
\end{minted}

\subsection{Unit tests}
\label{appx:testing:unit_tests}
A unit test should test the smallest unit of the program, typically a single function. While we could load each C function from our library separately and test it, we have instead opted for unit testing the Python functions acting as interfaces to the underlying C library. Such testing is typically referred to as \emph{black box testing} -- we make no assumptions about the specific implementation, we only study the function from what output is yielded for various input.

The tests we will be performing are simple to write, yet they are able prove that our algorithm is correct.

If the algorithms satisfy the following properties, we can be fairly sure that our \emph{algorithm} is correct. There may still be programmatic errors such as memory leaks and bad type handling, but those are difficult to test without from a black box perspective.
\begin{enumerate}[label=(\roman*)]
    \item Running a single iteration of the algorithm yields new values $v^{l+1}$ which are related to the previous values $v^l$ as prescribed by the selected scheme. \label{lst:prop1}
    \item Running a single iteration two times, yield the same answer as running two iterations in one function call.\label{lst:prop2}
\end{enumerate}

Notice that this list of properties have the structure of a proof by induction. We first prove that our algorithm is correct for a single step $l = 1$, and then that running a single step after $l = n$ yields $l = n+1$.

While property \ref{lst:prop1} is a mandatory test to perform, it may be less obvious why property \ref{lst:prop2} is relevant from a programmatic point of view. Here we should note that our implementation of the algorithm works by storing a pair of arrays containing $v^l$ and $v^{l+1}$ at any given time step $l$. Instead of always writing into the array for $v^{l+1}$ and then copying its contents into $v^l$ at the end of the time step, we swap their pointers to avoid unnecessary memory writes. It is easy to make errors when performing such pointer swaps, but if we test with both an odd and an even number of iterations, we should be pretty confident that our implementation is correct.

These tests can be found in \program{src/tests.py} and they can be run with

\begin{minted}[bgcolor=cbg_blue1]{text}
$ make pytest
\end{minted}

\subsection{Doctests}
\label{appx:testing:doctests}
We have written a few doctests which are more of the sort where we look at the result and consider whether it is reasonable or not.

These tests can be found in \program{src/doctests.py} and they can be run with

\begin{minted}[bgcolor=cbg_blue1]{text}
$ make pytest
\end{minted}


\subsection{Testing for memory related errors with Valgrind}
\label{appx:testing:valgrind}
We wrote some minimal C programs which call our solvers so that we could run them with Valgrind to discover memory leaks and other memory related errors which may be hard to detect.

A particularly useful feature of Valgrind is that it can discover errors with uninitialised values, so we have designed our tests to initialise all variables properly and use the resulting $v$ array to compute a sum of all elements. Then we have an if statement which needs to compare that sum to a particular value. If not all elements of $v$ are properly initialised, this will result in Valgrind detecting a conditional jump depending on an uninitialised value.

These tests can be run from the root directory of this project with:

\begin{minted}[bgcolor=cbg_blue1]{text}
$ make memtest
\end{minted}

\subsection{Automating the testing with Travis CI}
\label{appx:testing:travis}
Travis CI is a web service which allows automated testing of software. It is integrated with GitHub in such a way that when we push to our repository, a build job is scheduled on Travis. An overview of the latest build status is available at \url{https://travis-ci.org/KGHustad/FYS3150}.

Travis allows us to perform regular and reproducible tests and logs the results for us. Due to its integration with GitHub, the \href{https://github.com/KGHustad/FYS3150/commits/master}{list of commits for our repository} shows a green tick or a red cross depending on whether the build passed or failed, respectively.

Moreover, one can test different combinations of software to ensure that it works for more than one configuration, so in principle, we could have tested our programs with Python 2.6, 2.7, 3.5, etc., however, we have just tested with Python 2.7 since that is the version we have been using.

We have configured Travis to run all the aforementioned tests in addition to compiling all the C code from this project as well as previous projects.

%\newpage
%\section{Alternative applications of the diffusion equation -- Project 4 revisited}
\label{appx:alternative_applications}

In this brief appendix we an alternative application of the diffusion equation to reduce noise in a plot so that the peak can be found.

\subsection{A glimpse into the past}
In project 4 \cite{hustad_project4}, we had a very noisy plot where we needed to find the value along the x-axis where the plot peaked. The data we plotted had taken 360 CPU hours to compute, so, being responsible and considerate persons, we were hesitant to start another run with increase precision, taking up a significant portion of the limited computing resources offered by the University of Oslo.

However, we found a pair of servers which are not in regular use and reran the computation with 5 times higher precision with the workload distributed between them. The results were less noisy, but still not as smooth as we would have wanted.

\subsection{Looking at the smoothed figures}
\begin{figure}[H]
\includegraphics[width=\textwidth]{fig/{project4_smoothed_task_e_dT=0.001_sweeps=2E+06_2016-11-15--18-13-00}.pdf}
\caption{}
\label{fig:appx:alternative_applications:2e6}
\end{figure}


\begin{figure}[H]
\includegraphics[width=\textwidth]{fig/{project4_smoothed_joined}.pdf}
\caption{}
\label{fig:appx:alternative_applications:1e7}
\end{figure}


\end{document}
