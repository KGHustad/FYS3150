\documentclass[10pt,a4paper]{article}

%__Fonts and layout__
\usepackage[utf8]{inputenc}	%Allows input of unusual characters
\usepackage[T1]{fontenc,url}	%Helps correctly display unusual characters
\usepackage{parskip}	%Something about paragraph spacing
\usepackage{lmodern}	%Makes your font prettier or something
\usepackage{microtype}	%Makes text look prettier
\usepackage{verbatim}

%__math__
\usepackage{amsmath, amssymb}	%Improved math syntax and symbols
\usepackage{mathtools}	%Some matrix stuff and shit
\usepackage{tikz}	%Graph drawing
\usepackage{physics}	%Mathematical physics notation

%__algorithms and programming__
\usepackage{algorithm}	%Allows writing of pretty algorithm. Perfect for desplaying code-syntax
\usepackage{algpseudocode}	%Similar to algorithm, just different layout
\usepackage{listings}	%Listing-enviroment, perfect for code or terminal output
\usepackage{enumerate}	%For listing of stuff

%__graphs and pictures__
\usepackage{graphicx}	%Includegraphics
\usepackage{float}	%Allows the [H] option, to force graphics in place

%__quotes and refrencing__
\usepackage{epigraph}	%Allows epigraph enviroment, for quotes
\usepackage{hyperref}	%Allows hyperreferences in pdf
\usepackage[backend=biber]{biblatex}	%Citations
\addbibresource{cites.bib}	%Reference to citation file


\begin{document}



%__Making a first-page__
\title{This is the title}
\author{
	\begin{tabular}{rl}
		Author nr 1 & (\textit{username1})\\
		Author nr 2 & (\textit{username2})\\
	\end{tabular}}
\date{01.01.2000}
\maketitle



%__Epigraph__
\setlength{\epigraphwidth}{0.75\textwidth}
\renewcommand{\epigraphflush}{center}
\renewcommand{\beforeepigraphskip}{50pt}
\renewcommand{\afterepigraphskip}{100pt}
\renewcommand{\epigraphsize}{\normalsize}
\epigraph{This is a quote}
	{\textit{By this guy}}



%__Abstract__
\begin{abstract}
\noindent
This is an abstract
\end{abstract}
\pagebreak



%__Label and refrencing__
\section{Introduction}\label{Int}
This is a reference to the Introduction\ref{Int}



%__Cites__
This is a cite\cite{kim2016design}
\printbibliography %Prints citations. Should be at the bottom of document
\pagebreak


%__Math__
%Units and such should usually not be in italic.
\begin{equation}\begin{split}
4\text{ m} = 400\text{ mm}
\end{equation}\end{split}


%__Algorithm__
\begin{algorithm}
\caption{Euler-Cromer} \label{alg:euler_cromer}
\begin{algorithmic}[1]
  \Procedure{EulerCromer}{$p, v, p\, v, m, n$}
    \For {$i \gets 0, \dots, n-1$}
        \State $acc \gets $ \textsc{Acceleration}$(p, v_{i}, m, i, n, dt)$
        \State $v \gets v_{i} + acc$
        \State $p \gets p_{i}$
    \EndFor
  \EndProcedure
\end{algorithmic}
\end{algorithm}
\pagebreak



%__Listings__
\begin{lstlisting}[basicstyle=\footnotesize, frame=single, caption = This is a list caption, label = lst:list1]
this	is
a	list
:)
\end{lstlisting}
\pagebreak



%__Graphics___
\begin{figure}[H]
\centering	%Makes the graphic always stay in the center, no matter it's size.
\includegraphics[width=0.4\textwidth]{fig/art.png}
\includegraphics[width=0.3\textwidth]{fig/art.png}
\caption{This is a figure-caption}
\label{fig:figure1}
\end{figure}
This is a reference\ref{fig:figure1} to a figure.
\pagebreak



%__Footnotes__
This is a text with a footnote\footnote{This is a footnote}



%__EndNotes
\newpage	%Begins a new page and leaves everything as is
\pagebreak	%Begins a new page and spreads everything on the last page evenly out
\vfill		%It puts some empty space of elastic size
\newcommand{\dt}{\Delta t}	%Makes a new command "\dt", which calls "\Delta t"
\end{document}
