\documentclass[10pt,a4paper]{article}

\usepackage[utf8]{inputenc}
\usepackage[T1]{fontenc,url}
\usepackage{parskip}
\usepackage{lmodern}
\usepackage{microtype}
\usepackage{verbatim}
\usepackage{amsmath, amssymb}
\usepackage{mathtools}
\usepackage{tikz}
\usepackage{physics}
\usepackage{algorithm}
\usepackage{algpseudocode}
\usepackage{listings}
\usepackage{enumerate}
\usepackage{graphicx}
\usepackage{float}
\usepackage{epigraph}
\usepackage{hyperref}


\newcommand{\indexset}{\mathcal{I}}
\newcommand{\indexsetinner}{\mathcal{I}_{\mathrm{inner}}}
\newcommand{\bigO}{{\mathcal{O}}}
\newcommand{\bigtheta}{\mathcal{\Theta}}
\newcommand{\half}{\frac{1}{2}}
\newcommand{\dt}{{\Delta t}}
\newcommand{\dx}{{\Delta x}}


\begin{document}



\title{FYS3150 -- Project 5 -- PDE}
\author{
	\begin{tabular}{rl}
        Kristian Gregorius Hustad & (\texttt{krihus})\\
        Jonas Gahr Sturtzel Lunde & (\texttt{jonassl})
	\end{tabular}}
\date{December 9, 2016}
\maketitle



\setlength{\epigraphwidth}{0.75\textwidth}
\renewcommand{\epigraphflush}{center}
\renewcommand{\beforeepigraphskip}{50pt}
\renewcommand{\afterepigraphskip}{100pt}
\renewcommand{\epigraphsize}{\normalsize}
\epigraph{Nobody reads my lecture notes}
	{\textit{Morten Hjorth-Jensen}}

\begin{abstract}
\noindent
This is an abstract
\end{abstract}

\pagebreak




\textbf{TODO}: Remember to cite \cite{hpl_fdm}!

\section{Introduction}
We will be solving the partial differential equation 
\begin{equation}\label{eqn:PDE}
\frac{\partial u(x_i,t_j)^2}{\partial x^2} = \frac{\partial u(x_i,t_j)}{\partial t}
\end{equation}

\section{Method and Idea}
\subsection{One dimmentional case}
\subsubsection{Disrectization}
\begin{equation}
u(x+\Delta x, t+\Delta t) = u(x_{i+1},t_{i+1}) = u_{i+1,j+1}
\end{equation}

\subsubsection{Forward Euler - Explicit scheme}
We will be deriving the explicit scheme by Taylor expansion around $u(x,t)$.
\begin{equation}
u(x,t+\Delta t) = u(x,t) + \frac{\partial u(x,t)}{\partial t} + O(\Delta t^2)
\end{equation}

\begin{equation}
\frac{\partial u(x,t)}{\partial t} = \frac{u(x,t+\Delta t) - u(x,t)}{\Delta t} - \frac{O(\Delta t^2)}{\Delta t}
\end{equation}


Approximating the partial differential equation \ref{eqn:PDE} with the explicit scheme, we get
\begin{equation}
\frac{u(x_i + \Delta x, t_j) - 2u(x_i,t_j) + u(x_i - \Delta x, t_j)}{\Delta x^2}
= \frac{u(x_i, t_j + \Delta t) - u(x_i, t_i)}{\Delta t}
\end{equation}

which we can discretize as
\begin{equation}
\frac{u_{i+1,j} - 2u_{i,j}+u_{i-1,j}}{\Delta x^2} = \frac{u_{i,j+1} - u_{i,j}}{\Delta t}
\end{equation}

which gives
\begin{equation}
u_{i,j+1} = \alpha u_{i-1,j} + (1-2\alpha)u_{i,j} + \alpha u_{i+1,j}
\end{equation}
where $\alpha = \frac{\Delta t}{\Delta x^2}$ is a known constant.


\subsubsection{Backward Euler - Implicit scheme}
\begin{align}
\frac{u_{i+1,j} - 2u_{i,j}+u_{i-1,j}}{\Delta x^2} &= \frac{u_{i,j} - u_{i,j-1}}{\Delta t} \\
u_{i,j-1} = \alpha u_{i+1,j} + (1 - 2\alpha )u_{i,j} + \alpha u_{i-1,j}
\end{align}

we define
\[
 = \begin{bmatrix}
1+2\alpha & -\alpha & 0 & \cdots & \cdots \\
-\alpha & 1+2\alpha & -\alpha & \cdots & \vdots \\
0 & \ddots & \ddots & \ddots & \vdots \\
\vdots & \ddots & \ddots & \ddots & -\alpha \\
\vdots & \cdots & \cdots & -\alpha & 1+2\alpha \\
\end{bmatrix}
\]
and
\[ V_j = \begin{bmatrix}
u_{0,j}\\
u_{1,j}\\
\cdots\\
u_{n-1,j}\\
u_{n,j}\\
\end{bmatrix}
\]
and get
\begin{equation}
ÂV_j = V_{j-1}
\end{equation}
which means we can define the state at time $j$, $V_j$ as
\begin{equation}
V_j = Â^{-j}V_0
\end{equation}


\subsubsection{The Crank-Nicolson scheme}


\subsubsection{The truncation error}


\subsection{Two dimmentional case}
\begin{equation}
\left(\frac{\partial^2 u(x,y,t)}{\partial x^2} + \frac{\partial^2 u(x,y,t)}{\partial y^2}\right) = \frac{\partial u(x,y,t)}{\partial t}
\end{equation}
Forward Euler
\begin{equation}
\frac{u_{i+1,j}^l - 2u_{i,j}^l + u_{i-1,j}^l}{h^2} + \frac{u_{i,j+1}^l - 2u_{i,j}^l + u_{i,j-1}^l}{h^2} = \frac{u_{i,j}^{l+1} - u_{i,j}^l}{\Delta t}
\end{equation}

\begin{equation}
u_{i,j}^{l+1} = u_{i,j}^l + \alpha\left( u_{i+1,j}^l + u_{i-1,j}^l + u_{i,j+1}^l + u_{i,j+1}^l - 4u_{i,j}^l \right)
\end{equation}



\section{Implementation and results}\label{sec:implementation_and_results}

\subsection{Parallelization of the 2D explicit scheme}
\subsubsection{Work distribution and granularity}
\cite{inf3380_bok}
With the explicit scheme, we can compute $v_{ij}^l$ for all $i, j \in \indexsetinner$, if all values at the previous time step are known.

Since there are $(N - 1)^2$ independent values to be computed, the finest decomposition we could make, would be a 2D decomposition with $(N - 1)^2$ workers. Such a decomposition would, however, be terribly inefficient due to the overhead incurred from the needed communication.

If one assumes that the values have already been distributed among the $\frac{(N-1)^2}{n^2}$ workers, and that each worker is responsible for computing $n^2$ values. This would lead to each worker having to communicate $4n$ values to its neighbour workers at each time step. If the further assume that the workers are connected in a 2D mesh network, so that each worker is directly connected to its 4 neighbours, then we see that we have the communication at each time step runs in $\bigtheta (n)$ time.

In order to simplify matters, we use a 1D decomposition (along the y-axis).


\bibliographystyle{IEEEtran}
\bibliography{../papers}{}

\end{document}
